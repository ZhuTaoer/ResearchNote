% ========== 颜色 ==========
\definecolor{DefColor}{RGB}{0, 128, 128}        % 定义,青色
\definecolor{AssumpColor}{RGB}{128, 0, 128}     % 假设,紫色
\definecolor{TheoremColor}{RGB}{139, 0, 0}      % 定理、引理、命题,深红
\definecolor{ExampleColor}{RGB}{255, 140, 0}    % 例子,橙色
\definecolor{ProofColor}{RGB}{80, 80, 80}       % 证明,灰色
\definecolor{NoteColor}{RGB}{70, 130, 180}      % 笔记,钢蓝色

% ========== 高亮 ==========
\definecolor{HighlightYellow}{RGB}{255, 255, 153}   % 浅黄色高亮
\definecolor{HighlightGreen}{RGB}{204, 255, 204}    % 浅绿色高亮
\definecolor{HighlightBlue}{RGB}{204, 229, 255}     % 浅蓝色高亮
\definecolor{HighlightPink}{RGB}{255, 204, 229}     % 浅粉色高亮
\definecolor{HighlightOrange}{RGB}{255, 229, 204}   % 浅橙色高亮

% 基础高亮命令(默认黄色)
\newcommand{\hl}[1]{\colorbox{HighlightYellow}{#1}}

% 多种颜色的高亮命令
\newcommand{\hly}[1]{\colorbox{HighlightYellow}{#1}}      % 黄色
\newcommand{\hlg}[1]{\colorbox{HighlightGreen}{#1}}       % 绿色
\newcommand{\hlb}[1]{\colorbox{HighlightBlue}{#1}}        % 蓝色
\newcommand{\hlp}[1]{\colorbox{HighlightPink}{#1}}        % 粉色
\newcommand{\hlo}[1]{\colorbox{HighlightOrange}{#1}}      % 橙色

% 带文字颜色的高亮(高亮 + 文字颜色)
\newcommand{\hlr}[1]{\colorbox{HighlightYellow}{\textcolor{red}{#1}}}  % 红字黄底
\newcommand{\hlbf}[1]{\colorbox{HighlightYellow}{\textbf{#1}}}         % 加粗黄底

% ========== 定义 ==========
\newtcbtheorem[number within=section]{definition}{定义}{
  enhanced,
  frame empty, interior empty,
  colframe=DefColor!50!white,
  coltitle=DefColor!90!white,
  colbacktitle=DefColor!10!white,
  fonttitle=\bfseries,
  borderline={0.5mm}{0pt}{DefColor!10!white},
  borderline={0.5mm}{0pt}{DefColor!50!white,dashed},
  attach boxed title to top left={yshift=-2mm, xshift=2mm},
  boxed title style={boxrule=0.4pt},
  varwidth boxed title,
  breakable
}{def}

% ========== 假设 ==========
\newtcbtheorem[number within=section]{assumption}{假设}{
  enhanced,
  frame empty, interior empty,
  colframe=AssumpColor!50!white,
  coltitle=AssumpColor!90!white,
  colbacktitle=AssumpColor!10!white,
  fonttitle=\bfseries,
  borderline={0.5mm}{0pt}{AssumpColor!10!white},
  borderline={0.5mm}{0pt}{AssumpColor!50!white,dashed},
  attach boxed title to top left={yshift=-2mm, xshift=2mm},
  boxed title style={boxrule=0.4pt},
  varwidth boxed title,
  breakable
}{asm}

% ========== 定理 ==========
\newtcbtheorem[number within=section]{theorem}{定理}{
  enhanced,
  frame empty, interior empty,
  colframe=TheoremColor!50!white,
  coltitle=TheoremColor!90!white,
  colbacktitle=TheoremColor!10!white,
  fonttitle=\bfseries,
  borderline={0.5mm}{0pt}{TheoremColor!10!white},
  borderline={0.5mm}{0pt}{TheoremColor!50!white,dashed},
  attach boxed title to top left={yshift=-2mm, xshift=2mm},
  boxed title style={boxrule=0.4pt},
  varwidth boxed title,
  breakable
}{thm}

% ========== 引理 ==========
\newtcbtheorem[number within=section]{lemma}{引理}{
  enhanced,
  frame empty, interior empty,
  colframe=TheoremColor!50!white,
  coltitle=TheoremColor!90!white,
  colbacktitle=TheoremColor!10!white,
  fonttitle=\bfseries,
  borderline={0.5mm}{0pt}{TheoremColor!10!white},
  borderline={0.5mm}{0pt}{TheoremColor!50!white,dashed},
  attach boxed title to top left={yshift=-2mm, xshift=2mm},
  boxed title style={boxrule=0.4pt},
  varwidth boxed title,
  breakable
}{lem}

% ========== 命题 ==========
\newtcbtheorem[number within=section]{proposition}{命题}{
  enhanced,
  frame empty, interior empty,
  colframe=TheoremColor!50!white,
  coltitle=TheoremColor!90!white,
  colbacktitle=TheoremColor!10!white,
  fonttitle=\bfseries,
  borderline={0.5mm}{0pt}{TheoremColor!10!white},
  borderline={0.5mm}{0pt}{TheoremColor!50!white,dashed},
  attach boxed title to top left={yshift=-2mm, xshift=2mm},
  boxed title style={boxrule=0.4pt},
  varwidth boxed title,
  breakable
}{prop}

% ========== 例子 ==========
\newtcbtheorem[number within=section]{example}{例子}{
  enhanced,
  frame empty, interior empty,
  colframe=ExampleColor!50!white,
  coltitle=ExampleColor!90!black,
  colbacktitle=ExampleColor!15!white,
  fonttitle=\bfseries,
  borderline={0.5mm}{0pt}{ExampleColor!15!white},
  borderline={0.5mm}{0pt}{ExampleColor!50!white,dashed},
  attach boxed title to top left={yshift=-2mm, xshift=2mm},
  boxed title style={boxrule=0.4pt},
  varwidth boxed title,
  breakable
}{ex}

% ========== 证明 ==========
\tcolorboxenvironment{proof}{
  blanker,
  breakable,
  left=5mm,
  before skip=10pt,
  after skip=10pt,
  borderline west={1mm}{0pt}{ProofColor!50!white}
}

% ========== 笔记 ==========
\newtcbtheorem[number within=section]{note}{笔记}{
  enhanced,
  frame empty, interior empty,
  colframe=NoteColor!50!white,
  coltitle=NoteColor!90!white,
  colbacktitle=NoteColor!10!white,
  fonttitle=\bfseries,
  borderline={0.5mm}{0pt}{NoteColor!10!white},
  borderline={0.5mm}{0pt}{NoteColor!50!white,dashed},
  attach boxed title to top left={yshift=-2mm, xshift=2mm},
  boxed title style={boxrule=0.4pt},
  varwidth boxed title,
  breakable
}{note}